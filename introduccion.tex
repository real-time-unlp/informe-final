\section{Introducción}
Los semáforos son dispositivos utilizados para gestionar el tráfico, permitiendo o impidiendo el paso de conductores y peatones en cada vía. En gran cantidad de países, los semáforos son sistemas temporizados
que pasan de un estado a otro siguiendo un patrón de secuencia fija, careciendo de inteligencia para tomar decisiones. Ésto representa una gran desventaja durante las horas picos ya que los cambios se realizan en tiempos no adaptados a las condiciones del tráfico. En muchos casos ocurre que mientras una intersección vacía tiene luz verde, la calle principal se detiene a esperar el cambio, agrupando los vehículos hasta congestionar el canal.
Esta falta de sincronización entre las distintas intersecciones hacen que los vehículos deban arrancar y frenar constantemente a la llegada de cada cruce. Esta acción impacta negativamente sobre el medioambiente ya que aumenta el consumo de combustible y las emisiones de dióxido de carbono.

Son muchos los países que han intentado brindar una solución a esta problemática utilizando diferentes métodos. Es el caso de Brasil dónde diferentes ciudades utilizan el sistema de Google Maps para determinar dónde hay más vehículos y en función de dicha información, prolongar la onda verde según corresponda.
Por otra parte, en Róterdam, Holanda, las calles principales son equipadas con sensores de temperatura que detectan el grado de calor derivado del número de bicis que se acercan y en función de ese valor, prolonga o acorta la señal de paso.

Este proyecto busca implementar un sistema de semáforos inteligentes con el objetivo de disminuir la frecuencia con que suceden estos problemas. Además, se tiene como objetivo investigar la viabilidad del uso del planificador del sistema operativo de tiempo real para sistemas embebidos FreeRTOS, diseñando distintos algoritmos con sus implementaciones correspondientes y analizando la complejidad, extensibilidad y mantenibilidad de cada solución.