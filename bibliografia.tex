\section{Bibliograía}\label{sec:bib}

En éste proyecto se utilizó una implementación no oficial para Arduino UNO, el cual se encuentra en un repositorio git (ver \href{https://github.com/feilipu/Arduino\_FreeRTOS\_Library}{link}) que ha sido creado para proporcionar acceso a las capacidades de FreeRTOS.

Por otro lado, a continuación se detallan enlaces útiles que se tuvieron referencia en el informe:
\begin{itemize}
    \item SEMINT: Semaforos en brasil. \href{https://www.youtube.com/channel/UCLH5rWVe1ScpCVoa1rcqT7A}{link}.
    \item Holanda: LEDs para los peatones \href{http://www.t13.cl/noticia/tendencias/ciudad-holandesa-instala-semaforos-suelo-peatones-van-distraidos-sus-celulares}{link}.
    \item Holanda: Sensores para detectar bicicletas \href{https://bicycledutch.wordpress.com/2015/06/09/bicycle-parking-guidance-system-in-utrecht/}{link}.
    \item Motivaciones: \href{http://iiste.org/Journals/index.php/CEIS/article/view/37759}{Semaforos inteligentes basado en sensores},
    \href{http://hyclassproject.com/design-and-construction-of-an-intelligent-traffic-light-control-system.html}{Diseño}.
\end{itemize}